\documentclass[fleqn]{article}
\usepackage[left=1in, right=1in, top=1in, bottom=1in]{geometry}
\usepackage{mathexam}
\usepackage{amsmath}
\usepackage{amsfonts}

\ExamClass{Algebra}
\ExamName{Secondo Appello}
\ExamHead{13 Febbraio 2023}

\let\ds\displaystyle

\begin{document}
\begin{enumerate}
    \item Risolvere (se possibile) i seguenti sistemi di congruenze:
   
    \begin{enumerate} 
       \item \begin{center} 
       $ \begin{cases}
          10x \equiv 14 \pmod{18} \\
          16x \equiv 8 \pmod{22} \\
          9x \equiv 24 \pmod{30} 
       \end{cases} $
       \end{center}
       \item \begin{center} 
       $ \begin{cases}
          x \equiv 7 \pmod{15} \\
          x \equiv 12 \pmod{20} \\
          x \equiv 10 \pmod{24} 
       \end{cases} $
       \end{center}
       \item \begin{center}
       $ \begin{cases}
          x \equiv 6 \pmod{15} \\
          x \equiv 15 \pmod{18} \\
          x \equiv 1 \pmod{20} 
       \end{cases} $
       \end{center}
    \end{enumerate}

    \item Discutere il comportamento del sistema(ovvero, se è determinato, indeterminato o incompatibile)
    \begin{center}
    $\begin{cases} 
       kx + y + z = 1 \\
       x + ky + z  = k \\
       x + y + kz = k^2
    \end{cases}$ \\
    \end{center}
    Al variare del parametro $k \in \mathbb{R}$. Per quei valori di k che rendono il sistema determinato,
    calcolare l'unica soluzione. Per quei valori di k che rendono il sistema indeterminato, determinare
    una rappresentazione parametrica dell'insieme delle soluzioni.

    \item Calcolare l'ordine e l'inverso di $[223]$ nel gruppo $\mathbb{Z}_{360}^*$.
    
    \item Stabilire per quali valori del parametro $k \in \mathbb{R}$ la matrice 
    \begin{center}
    $A = \begin{pmatrix}
          3 & k & 3 \\
          0 & 0 & -k \\
          0 & 1-k & 1 
       \end{pmatrix}$
    \end{center}
    è diagonalizzabile (sul campo $\mathbb{R}$ dei numeri reali).

    \item Sia $A_n \in Mat_{n \times n}(\mathbb{R})$ la matrice con tutti uno sulla prima riga, la prima colonna
    e la diagonale principale, e tutti zero altrove. Ad esempio per $n=2,3,4,5,...$
    \begin{center}
    $
    A_2 = \begin{pmatrix}
          1 & 1 \\
          1 & 1
       \end{pmatrix},
    A_3 = \begin{pmatrix}
        1 & 1 & 1\\
        1 & 1 & 0  \\
        1 & 0 & 1 
     \end{pmatrix},
    A_4 = \begin{pmatrix}
        1 & 1 & 1 & 1\\
        1 & 1 & 0 & 0  \\
        1 & 0 & 1 & 0 \\
        1 & 0 & 0 & 1 
     \end{pmatrix},
    A_5 = \begin{pmatrix}
        1 & 1 & 1 & 1 & 1\\
        1 & 1 & 0 & 0 & 0  \\
        1 & 0 & 1 & 0 & 0 \\
        1 & 0 & 0 & 1 & 0 \\
        1 & 0 & 0 & 0 & 1
     \end{pmatrix},
     ...
    $
    \end{center}
     Mostrare che det$(A_n) = 2-n$.
\end{enumerate}
Esame trascritto in \textbf{\LaTeX} da Lucian D. Crainic.
\end{document}