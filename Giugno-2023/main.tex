\documentclass[fleqn]{article}
\usepackage[left=1in, right=1in, top=1in, bottom=1in]{geometry}
\usepackage{mathexam}
\usepackage{amsmath}
\usepackage{amsfonts}

\ExamClass{Algebra}
\ExamName{Quarto Appello}
\ExamHead{16 Giugno 2023}

\let\ds\displaystyle

\begin{document}
\begin{enumerate}
   \item Risolvere (se possibile) i seguenti sistemi di congruenze:
   
   \begin{enumerate} 
      \item \begin{center} 
      $ \begin{cases}
         10x \equiv 6 \pmod{16} \\
         12x \equiv 18 \pmod{21} \\
         10x \equiv 16 \pmod{22} 
      \end{cases} $
      \end{center}
      \item \begin{center} 
      $ \begin{cases}
         x \equiv 15 \pmod{24} \\
         x \equiv 21 \pmod{30} \\
         x \equiv 33 \pmod{36} 
      \end{cases} $
      \end{center}
      \item \begin{center}
      $ \begin{cases}
         x \equiv 7 \pmod{24} \\
         x \equiv 25 \pmod{30} \\
         x \equiv 31 \pmod{36} 
      \end{cases} $
      \end{center}
   \end{enumerate}

   \item Discutere il comportamento del sistema(ovvero, se è determinato, indeterminato o incompatibile)
   \begin{center}
   $\begin{cases} 
      x + ay + bz = b \\
      bx + z = 0 \\
      x + ay + z = 2
   \end{cases}$ \\
   \end{center}
   al variare dei parametri $a,b \in \mathbb{R}$.

   \item \begin{enumerate}
      \item Trovare il massimo comune divisore dei seguenti polinomi $p(x),q(x) \in \mathbb{R}[x]$:
      \begin{center}
         $p(x)= x^5 + x^3 + x^2 + 1, \qquad q(x)=x^5-3x^4+2x^3+x^2-3x+2$.
      \end{center}
      \item Calcolare la decomposizione dei precedenti polinomi $p(x),q(x)$ nel prodotto di fattori irriducibili
      negli anelli $\mathbb{R}[x]$ e $\mathbb{C}[x]$.
   \end{enumerate}

   \item Si consideri la matrice (dipendente dal parametro $a \in \mathbb{R}$)
   \begin{center}
   $A = \begin{pmatrix}
         a & 1 & 1  \\
         1 & 2 & 3  \\
         -1 & -1 & -2 
      \end{pmatrix}
      \in Mat_{3 \times 3}(\mathbb{R}).$
   \end{center}
   \begin{enumerate}
      \item Stabilire per quali valori di $a$ la matrice è diagonalizzabile (sul campo $\mathbb{R}$ 
      dei numeri reali).
      \item Posto $a=0$ (e avendo mostrato al precedento punto (a) che A è diagonalizzabile per questa scelta
      di $a \in \mathbb{R}$), determinare una matrice invertibile $B \in GL(3,\mathbb{R})$ e una matrice diagonale 
      $D \in Mat_{3 \times 3}(\mathbb{R})$ tali che 
      \begin{center}
      $D=B^{-1}AB$.
      \end{center}
   \end{enumerate}

   \item Date applicazioni lineari $f:U \rightarrow V$ e $g:V \rightarrow W$ tra spazi vettoriali di 
   dimensione finita, $U,V$ e $W$, e donotando con $g \circ f : U \rightarrow W$ la loro composizione, mostrare che
   \begin{enumerate}
      \item $rk(g \circ f) \leq rk(f)$, e vale l'uguaglianza se e soltanto se
      \begin{center}
          Im$(f) \cap $ Ker$(g) = \{ 0_V \}$.
      \end{center}
      \item $rk(g \circ f) \leq rk(g)$, e vale l'uguaglianza se e soltanto se
      \begin{center}
         Im$(f) + $ Ker$(g) = V$.
      \end{center}
   \end{enumerate}

\end{enumerate}
Esame trascritto in \textbf{\LaTeX} da Lucian D. Crainic.
\end{document}