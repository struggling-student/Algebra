\documentclass[fleqn]{article}
\usepackage[left=1in, right=1in, top=1in, bottom=1in]{geometry}
\usepackage{mathexam}
\usepackage{amsmath}
\usepackage{amsfonts}

\ExamClass{Algebra}
\ExamName{Quinto Appello}
\ExamHead{14 Luglio 2024}

\let\ds\displaystyle

\begin{document}
\begin{enumerate}
   \item Risolvere (se possibile) i seguenti sistemi di congruenze:
   
   \begin{enumerate} 
      \item \begin{center} 
      $ \begin{cases}
         8x \equiv 2 \pmod{18} \\
         9x \equiv 12 \pmod{21} \\
         14x \equiv 10 \pmod{22} 
      \end{cases} $
      \end{center}
      \item \begin{center} 
      $ \begin{cases}
         x \equiv 5 \pmod{18} \\
         x \equiv 3 \pmod{20} \\
         x \equiv 11 \pmod{24} 
      \end{cases} $
      \end{center}
      \item \begin{center}
      $ \begin{cases}
         x \equiv 7 \pmod{18} \\
         x \equiv 13 \pmod{20} \\
         x \equiv 19 \pmod{24} 
      \end{cases} $
      \end{center}
   \end{enumerate}

   \item Discutere il comportamento del sistema(ovvero, se è determinato, indeterminato o incompatibile)
   \begin{center}
   $\begin{cases} 
      x-2y+3z = 4\\
      2x - 3y + az = 5\\
      3x - 4y + 5z  = b
   \end{cases}$ \\
   \end{center}
   al variare dei parametri $a,b \in \mathbb{R}$.

   \item Date le seguenti permutazioni nel gruppo simmetrico $S_9$:
   \begin{center}
      $ \sigma_{1} = \begin{pmatrix}
         1 & 2 & 3 & 4 & 5 & 6 & 7 & 8 & 9 \\
         5 & 9 & 7 & 8 & 3 & 2 & 1 & 4 & 6
      \end{pmatrix} $

      $ \sigma_{2} = \begin{pmatrix}
         1 & 2 & 3 & 4 & 5 & 6 & 7 & 8 & 9 \\
         3 & 8 & 6 & 9 & 5 & 1 & 4 & 7 & 2
      \end{pmatrix} $

      $ \sigma_{3} = \begin{pmatrix}
         1 & 2 & 3 & 4 & 5 & 6 & 7 & 8 & 9 \\
         2 & 6 & 5 & 9 & 3 & 7 & 1 & 4 & 8
      \end{pmatrix} $
   \end{center}

   \begin{enumerate}
      \item Calcolare il prodotto $\sigma_{1}^{-1} \sigma_{2}^{-1} \sigma_{3}^{-1}$ nel gruppo $S_9$.
      \item Calcolare il segno e l'ordine di $\sigma_{1}$,$\sigma_{2}$ e $\sigma_{3}$.
      \item Per ogni $1 \leq i < j \leq 3$, stabilire se $\sigma_{i}$ e $\sigma_{j}$ sono coniugate tra loro, e in caso affermativo esibire $\alpha \in S_9$ tale che $\sigma_{j}= \alpha \sigma_{i} \alpha^{-1}$.
   \end{enumerate}

   \item Stabilire se la matrice 
   \begin{center}
    $
    A = \begin{pmatrix}
       2 & -1 & 0 \\
       -1 & 1 & -1 \\
       0 & -1 & 2
    \end{pmatrix}
    $
    \end{center}
   è diagonalizzabile (sul campo $\mathbb{R}$ dei numeri reali). In caso affermativo, determinare una matrice invertibile
   $B \in GL(3,\mathbb{R})$ e una matrice diagonale $D \in Mat_{3 \times 3}(\mathbb{R})$ tali che  
   \begin{center}
      $D= B^{-1}AB.$
   \end{center}
   
   \item Si consideri la matrice $A \in Mat_{n \times n}(\mathbb{R})$ con tutte le entrate uguali a 1,
   \begin{center}
   $
   A = \begin{pmatrix}
      1 & 1 & \cdots & 1 \\
      1 & 1 & \cdots & 1 \\
      \vdots & \vdots & \ddots & \vdots \\
      1 & 1 & \cdots & 1 \\
      \end{pmatrix}.
   $
   \end{center}

   \begin{enumerate}
      \item Calcolare gli autovalori di $A$.
      \item Stabilire se $A$ è diagonalizzabile, e in caso affermativo, determinare una matrice invertibile
      $B \in GL(n,\mathbb{R})$ e una matrice diagonale $D \in Mat_{n \times n}(\mathbb{R})$ tali che  
      \begin{center}
         $D= B^{-1}AB.$
      \end{center} 
   \end{enumerate}
  
\end{enumerate}
Esame trascritto in \textbf{\LaTeX} da Lucian D. Crainic.
\end{document}