\documentclass[fleqn]{article}
\usepackage[left=1in, right=1in, top=1in, bottom=1in]{geometry}
\usepackage{mathexam}
\usepackage{amsmath}
\usepackage{amsfonts}

\ExamClass{Algebra}
\ExamName{Sesto Appello}
\ExamHead{06 Settembre 2023}

\let\ds\displaystyle

\begin{document}
\begin{enumerate}
   \item Risolvere (se possibile) i seguenti sistemi di congruenze:
   
   \begin{enumerate} 
      \item \begin{center} 
      $ \begin{cases}
         8x \equiv 10 \pmod{18} \\
         9x \equiv 15 \pmod{24} \\
         21x \equiv 18 \pmod{33} 
      \end{cases} $
      \end{center}
      \item \begin{center} 
      $ \begin{cases}
         x \equiv 19 \pmod{24} \\
         x \equiv 13 \pmod{30} \\
         x \equiv 25 \pmod{36} 
      \end{cases} $
      \end{center}
      \item \begin{center}
      $ \begin{cases}
         x \equiv 21 \pmod{24} \\
         x \equiv 3 \pmod{30} \\
         x \equiv 9 \pmod{36} 
      \end{cases} $
      \end{center}
   \end{enumerate}

   \item Discutere il comportamento del sistema(ovvero, se è determinato, indeterminato o incompatibile)
   \begin{center}
   $\begin{cases} 
      x+ay+a^2z=1-b \\
      (4a+1)x-y-(8a^2-1)z=b \\
      x+y+z = 0
   \end{cases}$ \\
   \end{center}
   al variare dei parametri $a,b \in \mathbb{R}$.

   \item Calcolare l'ordine e l'inverso di $[719]$ nel gruppo $\mathbb{Z}_{1155}^*$.

   \item Stabilire per quali valori del parametro $a \in \mathbb{R}$ la seguente matrice 
   \begin{center}
    $
    A = \begin{pmatrix}
       1 & a & 0 \\
       -2 & 1 & 2 \\
       0 & a & 1
    \end{pmatrix}
    \in Mat_{3 \times 3}( \mathbb{R}).
    $ 
    \end{center}
      è diagonalizzabile (sul campo $\mathbb{R}$ dei numeri reali).
   
   \item Siano $g$ e $h$ elementi di un gruppo finito G tali che $gh = hg$. Detti $o(g), o(h), o(gh)$ rispettivamente
   gli ordini degli elementi $g, h, gh$ nel gruppo G:
    \begin{enumerate}
      \item mostrare che $o(gh)$ divide $mcm(o(g), o(h))$;
      \item mostrare che se $MCD(o(g), o(h)) = 1$ allora $o(gh) = o(g)o(h)$.
    \end{enumerate}
\end{enumerate}
Esame trascritto in \textbf{\LaTeX} da Lucian D. Crainic.
\end{document}