\documentclass[fleqn]{article}
\usepackage[left=1in, right=1in, top=1in, bottom=1in]{geometry}
\usepackage{mathexam}
\usepackage{amsmath}
\usepackage{amsfonts}

\ExamClass{Algebra}
\ExamName{Terzo Appello}
\ExamHead{31 Marzo 2023}

\let\ds\displaystyle

\begin{document}
\begin{enumerate}
   \item Risolvere (se possibile) i seguenti sistemi di congruenze:
   
   \begin{enumerate} 
      \item \begin{center} 
      $ \begin{cases}
         8x \equiv 12 \pmod{18} \\
         15x \equiv 12 \pmod{21} \\
         14x \equiv 10 \pmod{22} 
      \end{cases} $
      \end{center}
      \item \begin{center} 
      $ \begin{cases}
         x \equiv 15 \pmod{36} \\
         x \equiv 21 \pmod{40} \\
         x \equiv 11 \pmod{75} 
      \end{cases} $
      \end{center}
      \item \begin{center}
      $ \begin{cases}
         x \equiv 23 \pmod{36} \\
         x \equiv 11 \pmod{40} \\
         x \equiv 41 \pmod{75} 
      \end{cases} $
      \end{center}
   \end{enumerate}

   \item Discutere il comportamento del sistema(ovvero, se è determinato, indeterminato o incompatibile)
   \begin{center}
   $\begin{cases} 
      x + ay + 3z = 1 \\
      y + az = 0 \\
      x + y + z = b \\
      2x + ay + az = 2
   \end{cases}$ \\
   \end{center}
   al variare dei parametri $a,b \in \mathbb{R}$.
   \item Denotiamo con $(\mathbb{Z}_n^*, \cdot)$ il gruppo moltiplicativo degli interi invertibili modulo $n$.
   \begin{enumerate}
    \item stabilire se i gruppi $(\mathbb{Z}_9^*, \cdot)$ e $(\mathbb{Z}_{14}^*, \cdot)$ sono isomorfi tra loro.
    \item stabilire se i gruppi $(\mathbb{Z}_{24}^*, \cdot)$ e $(\mathbb{Z}_{30}^*, \cdot)$ sono isomorfi tra loro.
   \end{enumerate}   
   \textbf{Suggerimento:} Calcolare l'ordine di ogni elemento nel gruppo.

   \item Stabilire se la matrice 
   \begin{center}
    $
    A = \begin{pmatrix}
       3 & -5 & -1 \\
       1 & -3 & -1 \\
       6 & -5 & -4
    \end{pmatrix}
    $
    \end{center}
   è diagonalizzabile (sul campo $\mathbb{R}$ dei numeri reali). In caso affermativo, determinare una matrice invertibile
   $B \in GL(3,\mathbb{R})$ e una matrice diagonale $D \in Mat_{3 \times 3}(\mathbb{R})$ tali che  
   \begin{center}
      $D= B^{-1}AB.$
   \end{center}
   
   \item Siano dati i polinomi $p_1(x),...,p_n(x) \in \mathbb{K}[x]$ (con $\mathbb{K}$ un campo) a due a due co-primi
   tra loro, ovvero tali che $MCD(p_i(x),p_j(x))=1$ per ogni $i \neq j$. Denotando con $P(x)=p_1(x)p_2(x)...p_n(x)$ 
   il loro prodotto, siano $P_1(x),...,P_n(x) \in \mathbb{K}[x]$ i polinomi definiti da

   $
   P_1(x) := \frac{P(x)}{p_1(x)}=p_2(x)p_3(x)...p_n(x),
   $ 
   
      \qquad \qquad \qquad \qquad $P_2(x) := \frac{P(x)}{p_2(x)}=p_1(x)p_3(x)...p_n(x),$ 

      \qquad \qquad \qquad \qquad \qquad \qquad \qquad $..., \quad P_n(x) := \frac{P(x)}{p_n(x)}=p_1(x)p_2(x)...p_{n-1}(x),$ 

   \begin{enumerate}
      \item Mostrare che $MCD(p_i(x),P_i(x))=1$ per ogni $i=1,...,n$.
      \item Mostrare che $MCD(P_1(x),..,P_n(x))=1$.
      \item Mostrare che dato un arbitrario $q(x) \in \mathbb{K}[x]$, esistono $q_1(x),...,q_n(x) \in \mathbb{K}[x]$ tali che
      \begin{center}
         $q(x)=q_1(x)P_1(x)+q_2(x)P_2(x)+...+q_n(x)P_n(x)$.
      \end{center}
   \end{enumerate}
\end{enumerate}
Esame trascritto in \textbf{\LaTeX} da Lucian D. Crainic.
\end{document}