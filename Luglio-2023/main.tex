\documentclass[fleqn]{article}
\usepackage[left=1in, right=1in, top=1in, bottom=1in]{geometry}
\usepackage{mathexam}
\usepackage{amsmath}
\usepackage{amsfonts}

\ExamClass{Algebra}
\ExamName{Quinto Appello}
\ExamHead{14 Luglio 2024}

\let\ds\displaystyle

\begin{document}
\begin{enumerate}
   \item Risolvere (se possibile) i seguenti sistemi di congruenze:
   
   \begin{enumerate} 
      \item \begin{center} 
      $ \begin{cases}
         20x \equiv 24 \pmod{28} \\
         21x \equiv 27 \pmod{30} \\
         12x \equiv 18 \pmod{39} 
      \end{cases} $
      \end{center}
      \item \begin{center} 
      $ \begin{cases}
         x \equiv 9 \pmod{28} \\
         x \equiv 37 \pmod{42} \\
         x \equiv 13 \pmod{72} 
      \end{cases} $
      \end{center}
      \item \begin{center}
      $ \begin{cases}
         x \equiv 9 \pmod{28} \\
         x \equiv 23 \pmod{42} \\
         x \equiv 35 \pmod{72} 
      \end{cases} $
      \end{center}
   \end{enumerate}

   \item Discutere il comportamento del sistema(ovvero, se è determinato, indeterminato o incompatibile)
   \begin{center}
   $\begin{cases} 
      (b-1)x + (a-1)y + (1-a)z = 1-b\\
      (a-b)x + ay - bz = 0 \\
      2x + (1-a)y + (a+b)z = 1 + a
   \end{cases}$ \\
   \end{center}
   al variare dei parametri $a,b \in \mathbb{R}$.

   \item Date le seguenti permutazioni nel gruppo simmetrico $S_9$:
   \begin{center}
      $ \sigma_{1} = \begin{pmatrix}
         1 & 2 & 3 & 4 & 5 & 6 & 7 & 8 & 9 \\
         8 & 6 & 1 & 7 & 4 & 9 & 5 & 3 & 2
      \end{pmatrix} $

      $ \sigma_{2} = \begin{pmatrix}
         1 & 2 & 3 & 4 & 5 & 6 & 7 & 8 & 9 \\
         5 & 6 & 8 & 7 & 4 & 3 & 1 & 9 & 2
      \end{pmatrix} $

      $ \sigma_{3} = \begin{pmatrix}
         1 & 2 & 3 & 4 & 5 & 6 & 7 & 8 & 9 \\
         6 & 7 & 9 & 1 & 8 & 3 & 5 & 2 & 4
      \end{pmatrix} $
   \end{center}

   \begin{enumerate}
      \item Calcolare il prodotto $\sigma_{1}^{-1} \sigma_{2}^{-1} \sigma_{3}^{-1}$ nel gruppo $S_9$.
      \item Calcolare il segno e l'ordine di $\sigma_{1}$,$\sigma_{2}$ e $\sigma_{3}$.
      \item Per ogni $1 \leq i < j \leq 3$, stabilire se $\sigma_{i}$ e $\sigma_{j}$ sono coniugate tra loro, e in caso affermativo esibire $\alpha \in S_9$ tale che $\sigma_{j}= \alpha \sigma_{i} \alpha^{-1}$.
   \end{enumerate}

   \item Stabilire per quali valori del parametro $a \in \mathbb{R}$ la matrice 
   \begin{center}
   $A = \begin{pmatrix}
         1 & 0 & a^2 \\
         0 & a & 0 \\
         1 & 0 & 1 
      \end{pmatrix}
      \in Mat_{3 \times 3}(\mathbb{R})$.
   \end{center}
   è diagonalizzabile (sul campo $\mathbb{R}$ dei numeri reali).

   \item Sia $A_n \in Mat_{n \times n}(\mathbb{R})$ la matrice con tutti uno sulla prima riga, la prima colonna
   e la diagonale principale, e tutti zero altrove. Ad esempio per $n=2,3,4,5,...$
   \begin{center}
   $
   A_2 = \begin{pmatrix}
         0 & 1 \\
         1 & 0
      \end{pmatrix},
   A_3 = \begin{pmatrix}
       0 & 1 & 1 \\
       1 & 0 & 1 \\
       1 & 1 & 0 
    \end{pmatrix},
   A_4 = \begin{pmatrix}
       0 & 1 & 1 & 1\\
       1 & 0 & 1 & 1  \\
       1 & 1 & 0 & 1 \\
       1 & 1 & 1 & 0 
    \end{pmatrix},
   A_5 = \begin{pmatrix}
       0 & 1 & 1 & 1 & 1\\
       1 & 0 & 1 & 1 & 1 \\
       1 & 1 & 0 & 1 & 1 \\
       1 & 1 & 1 & 0 & 1 \\
       1 & 1 & 1 & 1 & 0
    \end{pmatrix},
    ...
   $
   \end{center}

   \begin{enumerate}
      \item Mostrare che il polinomio caratteristico di $A_n$ è $p_{A_n}(x) = (x+1)^{n-1}(x-n+1)$.
      \item Stabilire se $A_n$ è diagonalizzabile (sul campo $\mathbb{R}$ dei numeri reali).
   \end{enumerate}

\end{enumerate}
Esame trascritto in \textbf{\LaTeX} da Lucian D. Crainic.
\end{document}