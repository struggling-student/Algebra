\documentclass[fleqn]{article}
\usepackage[left=1in, right=1in, top=1in, bottom=1in]{geometry}
\usepackage{mathexam}
\usepackage{amsmath}
\usepackage{amsfonts}

\ExamClass{Algebra}
\ExamName{Primo Appello}
\ExamHead{27 Gennaio 2023}

\let\ds\displaystyle

\begin{document}
\begin{enumerate}
   \item Risolvere (se possibile) i seguenti sistemi di congruenze:
   
   \begin{enumerate} 
      \item \begin{center} 
      $ \begin{cases}
         12x \equiv 21 \pmod{33} \\
         25x \equiv 20 \pmod{35} \\
         21x \equiv 33 \pmod{39} 
      \end{cases} $
      \end{center}
      \item \begin{center} 
      $ \begin{cases}
         x \equiv 5 \pmod{18} \\
         x \equiv 17 \pmod{21} \\
         x \equiv 23 \pmod{24} 
      \end{cases} $
      \end{center}
      \item \begin{center}
      $ \begin{cases}
         x \equiv 7 \pmod{18} \\
         x \equiv 13 \pmod{21} \\
         x \equiv 11 \pmod{24} 
      \end{cases} $
      \end{center}
   \end{enumerate}

   \item Discutere il comportamento del sistema(ovvero, se è determinato, indeterminato o incompatibile)
   \begin{center}
   $\begin{cases} 
      x + y + az = 1 \\
      x + z = 0 \\
      x + y + a^3z = 3 \\
      x + y + z = b
   \end{cases}$ \\
   \end{center}
   al variare dei parametri $a,b \in \mathbb{R}$.

   \item Data la matrice 
   \begin{center}
   $A = \begin{pmatrix}
         1 & 0 & 2 & 3 \\
         -3 & 1 & -5 & -2 \\
         1 & -2 & 1 & -1 \\
         1 & 1 & 2 & 0
      \end{pmatrix}
      \in Mat_{4 \times 4}(\mathbb{R})$
   \end{center}
   si consideri l'applicazione lineare associata $L_A:\mathbb{R}^4 \rightarrow \mathbb{R}^4:v \rightarrow Av$
   \begin{enumerate}
      \item Determinare basi dei sottospazio ker($L_A$) e Im$(L_A)\subset \mathbb{R}$.
      \item Determinare equazioni cartesiane del sottospazio Im$(L_A)$.
   \end{enumerate}

   \item Stabilire se la matrice
   \begin{center}
      $
      A = \begin{pmatrix}
         1 & 2 & -1 \\
         1 & 0 & 1 \\
         4 & -4 & 5
      \end{pmatrix}
   $
   \end{center}
   è diagonalizzabile (sul campo $\mathbb{R}$ dei numeri reali). In caso affermativo, determinare una matrice invertibile
    $B \in GL(3,\mathbb{R})$ e una matrice diagonale $D \in Mat_{3 \times 3}(\mathbb{R})$ tali che  

   \begin{center}
      $D= B^{-1}AB.$
   \end{center}
   
   \item Dati interi $m,n \geq 1,$ sia $d := MCD(m,n)$ il loro massimo comun divisore. Dati i polinomi $p(x) := x^m-1$
   e $q(x) := x^n-1$, mostrare che  $MCD(p(x),q(x)) = x^d-1$. \\
   (\textbf{Suggerimento:} mostrare prima che dati interi 
   $a \geq b \geq 1$, se $r$ è il resto della divisione di $a$ per $b$, allora $x^r-1$ è il resto della divisione
   di $x^a-1$ per $x^b-1$).
\end{enumerate}
Esame trascritto in \textbf{\LaTeX} da Lucian D. Crainic.
\end{document}